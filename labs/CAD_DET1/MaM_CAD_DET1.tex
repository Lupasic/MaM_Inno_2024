% \documentclass[aspectratio=169,notes]{beamer}
\documentclass[aspectratio=169]{beamer}
\usetheme[faculty=phil]{fibeamer}
\usepackage{polyglossia}
\setmainlanguage{english} %% main locale instead of `english`, you
%% can typeset the presentation in either Czech or Slovak,
%% respectively.
\setotherlanguages{russian} %% The additional keys allow
%%
%%   \begin{otherlanguage}{czech}   ... \end{otherlanguage}
%%   \begin{otherlanguage}{slovak}  ... \end{otherlanguage}
%%
%% These macros specify information about the presentation
\title[MaM]{Mechanics and Machines, CAD Details 1} %% that will be typeset on the
\subtitle{Intro to subject
\\ History of CAD \\ Solid modeling
         } %% title page.
\author{Oleg Bulichev}
%% These additional packages are used within the document:
\usepackage{ragged2e}  % `\justifying` text
\usepackage{booktabs}  % Tables
\usepackage{tabularx}
\usepackage{tikz}      % Diagrams
\usetikzlibrary{calc, shapes, backgrounds}
\usepackage{amsmath, amssymb}
\usepackage{url}       % `\url`s
\usepackage{listings}  % Code listings
% \usepackage{subfigure}
\usepackage{floatrow}
\usepackage{subcaption}
\usepackage{mathtools}
\usepackage{todonotes}
\usepackage{fontspec}
\usepackage{multicol}
\usepackage{pdfpages}
\usepackage{wrapfig}
\usepackage{animate}
\usepackage{booktabs}
\usepackage{multirow}
% \usepackage{graphicx}
\usepackage{colortbl}

\graphicspath{{resources/}}
\frenchspacing

\setbeamertemplate{caption}[numbered]
\usetikzlibrary{graphs}

% \usepackage[backend=biber,style=ieee,autocite=footnote]{biblatex}
% \addbibresource{biblio.bib}
% \DefineBibliographyStrings{english}{%
%   bibliography = {References},}

\newcommand{\oleg}[2][] {\todo[color=red, #1] {OLEG:\\ #2}}
\newcommand{\fbckg}[1]{\usebackgroundtemplate{\includegraphics[width=\paperwidth]{#1}}}%frame background

\usepackage[framemethod=TikZ]{mdframed}
\newcommand{\dbox}[1]{
\begin{mdframed}[roundcorner=3pt, backgroundcolor=yellow, linewidth=0]
\vspace{1mm}
{#1}
\vspace{1mm}
\end{mdframed}
}

\begin{document}
\setlength{\abovedisplayskip}{0pt}
\setlength{\belowdisplayskip}{0pt}
\setlength{\abovedisplayshortskip}{0pt}
\setlength{\belowdisplayshortskip}{0pt}

\fbckg{fibeamer/figs/title_page.png}
\frame[c]{\setcounter{framenumber}{0}
    \usebeamerfont{title}%
    \usebeamercolor[fg]{title}%
    \begin{minipage}[b][6.5\baselineskip][b]{\textwidth}%
        \textcolor{black}{\raggedright\inserttitle}
    \end{minipage}
    % \vskip-1.5\baselineskip

    \usebeamerfont{subtitle}%
    \usebeamercolor[fg]{framesubtitle}%
    \begin{minipage}[b][3\baselineskip][b]{\textwidth}
        \raggedright%
        \insertsubtitle%
    \end{minipage}
    \vskip.25\baselineskip
}
%   \frame[c]{\maketitle}

\fbckg{fibeamer/figs/common.png}

\note{\scriptsize \begin{itemize}
        \item \
    \end{itemize}}

\begin{frame}[c]{Class Activities}
\framesubtitle{}
    \centering \LARGE
    Triple repetition: lab, HW, Project
\end{frame}

\begin{frame}[c]{}
\framesubtitle{}
\centering \LARGE \textbf{Lab Goals} \\
To obtain the needed tools for solving the design part of the course project
\end{frame}

\begin{frame}[t]{Final Exam (Previous Year)}
\framesubtitle{}
    \begin{figure}[H]
        \begin{subfigure}{0.32\textwidth}
            \centering\includegraphics[height=6cm,width=1\textwidth,keepaspectratio]{resources/ex1.png}
            \caption*{CAD part}
            \label{fig:resources/ex1.png}
        \end{subfigure}
        \begin{subfigure}{0.32\textwidth}
            \centering\includegraphics[height=6cm,width=1\textwidth,keepaspectratio]{resources/ex2.png}
            \caption*{CAE part}
            \label{fig:resources/ex2.png}
        \end{subfigure}
        \begin{subfigure}{0.32\textwidth}
            \centering\includegraphics[height=6cm,width=1\textwidth,keepaspectratio]{resources/ex3.png}
            \caption*{Theory part}
            \label{fig:resources/ex3.png}
        \end{subfigure}
    \end{figure}
\end{frame}

\begin{frame}[t]{Computer Aided Design}
\framesubtitle{}
    \vspace{-0.6cm}
    \begin{figure}[H]
        \centering\includegraphics[height=6.5cm,width=1\textwidth,keepaspectratio]{resources/CADCAMCAE.png}
        \label{fig:resources/CADCAMCAE.png}
    \end{figure}
\end{frame}

\begin{frame}[t]{Computer Aided Design}
\framesubtitle{Types of modeling}
    \vspace{-0.6cm}
    \begin{figure}[H]
        \centering\includegraphics[height=6cm,width=1\textwidth,keepaspectratio]{resources/solidsurface.png}
        \label{fig:resources/solidsurface.png}
    \end{figure}
\end{frame}

\begin{frame}[t]{History of CAD}
\framesubtitle{}
    \begin{itemize}
        \item[60th] --- Theoretical studies of the possibility of solving design problems on the computer were carried out.
        \item[70th] --- Methods, algorithms and programs for solving individual tasks for different design stages were developed.
        \item[80th] --- CAD is being developed and improved. 3D modeling became more popular.
        \item[90th] --- Developers had finished formation of base concepts of CAD and unified data transfer between systems.
    \end{itemize}
\end{frame}

\begin{frame}[c]{}
    \framesubtitle{}
        \LARGE\centering
        \textbf{CAD benefits}\\
        \medskip
        Cheaper\\
        Safer \\
        Faster
    \end{frame}

\begin{frame}[c]{Popular CAD systems in Russia}
\framesubtitle{}
    \vspace{-0.6cm}
    \begin{figure}[H]
        \centering\includegraphics[height=6.5cm,width=1\textwidth,keepaspectratio]{resources/CADs.png}
        \label{fig:resources/CADs.png}
    \end{figure}
\end{frame}

\begin{frame}[t]{Siemens NX}
\framesubtitle{}
    \begin{columns}[T,onlytextwidth]
        \begin{column}{0.32\textwidth}
            \centering\textbf{Prof}
            \begin{itemize}
                \item All in one system (CAD,CAM,CAE,PDM)
                \item Free for students
                \item Can create a real aircraft in it
            \end{itemize}
        \end{column}
        \begin{column}{0.32\textwidth}
            \begin{figure}[H]
                \centering\includegraphics[height=6cm,width=1\textwidth,keepaspectratio]{resources/NXpic.png}
                % \caption{caption_name}
                \label{fig:resources/NXpic.png}
            \end{figure}
        \end{column}
        \begin{column}{0.32\textwidth}
            \centering\textbf{Cons}
            \begin{itemize}
                \item Complex system
                \item Not popular in small companies
            \end{itemize}
        \end{column}
    \end{columns}
\end{frame}

\begin{frame}[t]{Common usage of other systems for our tasks}
\framesubtitle{}
    \begin{itemize}
        \item If you need a good drawings. Make CAD anywhere, afterwards import to Kompas-3D.
        \item If you need Standard Component Library (SCL), use either Kompas, or Solid Edge, or \href{https://www.mcmaster.com/}{mcmaster}. Insert needed stuff in NX.
    \end{itemize}
\end{frame}

\begin{frame}[t]{Basics}
    \framesubtitle{Video + labs/CAD\_DET1/task\_data/video1.zip}
    \vspace{-0.6cm}
    \begin{figure}[H]
        \href{https://disk.yandex.ru/i/osJBruJtY83Crw}{
            \centering\includegraphics[height=6cm,width=1\textwidth,keepaspectratio]{resources/task_0.png}}
        % \caption{Click on a picture for a video}
        \label{fig:1}
    \end{figure}
\end{frame}

\begin{frame}[t]{Task 1: make CAD model of detail below}
\framesubtitle{}
    \vspace{-0.6cm}
    \begin{figure}[H]
        \centering\includegraphics[height=6.5cm,width=1\textwidth,keepaspectratio]{resources/task_1.png}
        \label{fig:resources/task_1.png}
    \end{figure}
\end{frame}

\begin{frame}[t]{Task 2: make an animated sketch}
    \framesubtitle{Video (repeat the figure, not video)}
    \vspace{-0.6cm}
    \begin{figure}[H]
        \href{https://youtu.be/KohY2-krw1I}{
            \centering\includegraphics[height=6cm,width=1\textwidth,keepaspectratio]{resources/task_2.png}}
        \label{fig:task_2.png}
    \end{figure}
\end{frame}

    \begin{frame}[c]{Task 3: make CAD model of detail below}
        \framesubtitle{}
            \vspace{-0.6cm}
            \begin{figure}[H]
                \begin{subfigure}{0.65\textwidth}
                    \centering\includegraphics[height=6cm,width=1\textwidth,keepaspectratio]{resources/task_31.jpg}
                    % \caption{capture1}
                    \label{fig:resources/task_31.jpg}
                \end{subfigure}
                \begin{subfigure}{0.32\textwidth}
                    \centering\includegraphics[height=6cm,width=1\textwidth,keepaspectratio]{resources/task_32.png}
                    % \caption{capture2}
                    \label{fig:resources/task_32.png}
                \end{subfigure}
            \end{figure}
            \textbf{Hint}: It can be solved making 2 equal sketches perpendicular to each other, extruding them and using “combine” command. 

        \end{frame} 

\fbckg{fibeamer/figs/last_page.png}
\frame[plain]{}
\end{document}