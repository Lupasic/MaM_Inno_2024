% \documentclass[aspectratio=169,notes]{beamer}
\documentclass[aspectratio=169]{beamer}
\usetheme[faculty=phil]{fibeamer}
\usepackage{polyglossia}
\setmainlanguage{english} %% main locale instead of `english`, you
%% can typeset the presentation in either Czech or Slovak,
%% respectively.
\setotherlanguages{russian} %% The additional keys allow
%%
%%   \begin{otherlanguage}{czech}   ... \end{otherlanguage}
%%   \begin{otherlanguage}{slovak}  ... \end{otherlanguage}
%%
%% These macros specify information about the presentation
\title[MaM]{Mechanics and Machines, Lecture 3} %% that will be typeset on the
\subtitle{Types of drives: kinematics, where to find other info
\\ Drives: friction, belts, chains, gears, universal, geneva,       \\ ballscrew  
         } %% title page.
\author{Oleg Bulichev}
%% These additional packages are used within the document:
\usepackage{ragged2e}  % `\justifying` text
\usepackage{booktabs}  % Tables
\usepackage{tabularx}
\usepackage{tikz}      % Diagrams
\usetikzlibrary{calc, shapes, backgrounds}
\usepackage{amsmath, amssymb}
\usepackage{url}       % `\url`s
\usepackage{listings}  % Code listings
% \usepackage{subfigure}
\usepackage{floatrow}
\usepackage{subcaption}
\usepackage{mathtools}
\usepackage{todonotes}
\usepackage{fontspec}
\usepackage{multicol}
\usepackage{pdfpages}
\usepackage{wrapfig}
\usepackage{animate}
\usepackage{booktabs}
\usepackage{multirow}
% \usepackage{graphicx}
\usepackage{colortbl}

\graphicspath{{resources/}}
\frenchspacing

\setbeamertemplate{caption}[numbered]
\usetikzlibrary{graphs}

% \usepackage[backend=biber,style=ieee,autocite=footnote]{biblatex}
% \addbibresource{biblio.bib}
% \DefineBibliographyStrings{english}{%
%   bibliography = {References},}

\newcommand{\oleg}[2][] {\todo[color=red, #1] {OLEG:\\ #2}}
\newcommand{\fbckg}[1]{\usebackgroundtemplate{\includegraphics[width=\paperwidth]{#1}}}%frame background

\usepackage[framemethod=TikZ]{mdframed}
\newcommand{\dbox}[1]{
\begin{mdframed}[roundcorner=3pt, backgroundcolor=yellow, linewidth=0]
\vspace{1mm}
{#1}
\vspace{1mm}
\end{mdframed}
}

\begin{document}
\setlength{\abovedisplayskip}{0pt}
\setlength{\belowdisplayskip}{0pt}
\setlength{\abovedisplayshortskip}{0pt}
\setlength{\belowdisplayshortskip}{0pt}

\fbckg{fibeamer/figs/title_page.png}
\frame[c]{\setcounter{framenumber}{0}
    \usebeamerfont{title}%
    \usebeamercolor[fg]{title}%
    \begin{minipage}[b][6.5\baselineskip][b]{\textwidth}%
        \textcolor{black}{\raggedright\inserttitle}
    \end{minipage}
    % \vskip-1.5\baselineskip

    \usebeamerfont{subtitle}%
    \usebeamercolor[fg]{framesubtitle}%
    \begin{minipage}[b][3\baselineskip][b]{\textwidth}
        \raggedright%
        \insertsubtitle%
    \end{minipage}
    \vskip.25\baselineskip
}
%   \frame[c]{\maketitle}

\fbckg{fibeamer/figs/common.png}

\note{\scriptsize \begin{itemize}
        \item \
    \end{itemize}}

\begin{frame}[c]{Goal of the lecture}
    \framesubtitle{}
    \LARGE
    \centering
    Make an overview of typical drives. \\ Give a hint how to work with it. \\ Explain how to find information about particular drive.
\end{frame}

\begin{frame}[t]{General info about drives}
    \framesubtitle{Video}
    \vspace{-0.6cm}
    \begin{figure}[H]
        \href{https://youtu.be/XOh8qfXDRIs}{
            \centering\includegraphics[height=6cm,width=1\textwidth,keepaspectratio]{drives_preview.jpg}}
        % \caption{Click on a picture for a video}
        \label{fig:drives_preview.jpg}
    \end{figure}
\end{frame}

\begin{frame}[t]{Universal Joint}
    \framesubtitle{Visualisation}
    \vspace{-0.5cm}
    \begin{figure}[H]
        \begin{subfigure}{0.49\textwidth}
            \centering\includegraphics[height=6cm,width=1\textwidth,keepaspectratio]{universal_kinematics.png}
        \end{subfigure}
        \begin{subfigure}{0.49\textwidth}
            \href{https://en.wikipedia.org/wiki/Universal_joint\#/media/File:Universal_joint.gif}{
                \centering\includegraphics[height=6cm,width=1\textwidth,keepaspectratio]{cardan_video_preview.png}}
        \end{subfigure}
    \end{figure}
\end{frame}


\begin{frame}[t]{Universal Joint}
    \framesubtitle{Types of universal joint}
    \begin{figure}[H]
        \begin{subfigure}{0.32\textwidth}
            \centering\includegraphics[height=6cm,width=1\textwidth,keepaspectratio]{universal_1.png}
            \caption*{Cardan}
        \end{subfigure}
        \begin{subfigure}{0.32\textwidth}
            \href{https://en.wikipedia.org/wiki/Constant-velocity_joint\#/media/File:Double_Cardan_Joint_(animated).gif}{
                \centering\includegraphics[height=6cm,width=1\textwidth,keepaspectratio]{universal_2_video_preview.png}}
            \caption*{Double cardan joint}
        \end{subfigure}
        \begin{subfigure}{0.32\textwidth}
            \href{https://gifyu.com/image/SqMnR}{
                \centering\includegraphics[height=6cm,width=1\textwidth,keepaspectratio]{shrus_video_preview.png}}
            \caption*{Constant-velocity universal ball joint}
        \end{subfigure}
    \end{figure}
\end{frame}

\begin{frame}[t]{Universal Joint}
    \framesubtitle{Drive kinematics}
    \begin{columns}[T,onlytextwidth]
        \begin{column}{0.59\textwidth}
            Angle relationship --- $\tan(\psi)=\tan(\psi')\cos(\beta)$ \\
            Angular velocities relationship --- $ \omega \cos(\beta) = \omega'(\sin^2(\psi) + \cos^2(\psi)\cos^2(\beta))$
        \end{column}
        \begin{column}{0.39\textwidth}
            \vspace{-0.9cm}
            \begin{figure}[H]
                \centering\includegraphics[height=6cm,width=1\textwidth,keepaspectratio]{universal_kinematics.png}
                % \caption{caption_name}
                \label{fig:universal_kinematics.png}
            \end{figure}
        \end{column}
    \end{columns}

\end{frame}

\begin{frame}[t]{Universal Joint}
    \framesubtitle{Features and facts}
    \begin{itemize}
        \item It's effective tool for transferring a torque for max 30 degrees.
        \item Constant-velocity universal ball joint (шрус) is not a small device and it's not easy to find it (it can be found as a car detail).
    \end{itemize}
\end{frame}

\begin{frame}[t]{Universal Joint}
    \framesubtitle{What can be interesting to find (queries)}
    \begin{enumerate}
        \item Correlation between velocities and angle between links in Universal joint
        \item Cardan dynamics
    \end{enumerate}
\end{frame}

\begin{frame}[t]{Universal Joint}
    \framesubtitle{Reference material}
    \begin{enumerate}
        \item \textbf{Other names}: cardan joint, Hooke's joint, кардан, универсальный шарнир
        \item \href{https://en.wikipedia.org/wiki/Universal_joint}{Universal joint (wiki)}
        \item \textit{"Теория механизмов и машин" Артоболевский И. И. 1988, pdf pages 168--172 }
        \item \href{https://elar.urfu.ru/bitstream/10995/102516/1/2-s2.0-85107367228.pdf}{Find U-joint parameters using quaternions}
        \item \href{https://www.researchgate.net/publication/257774799_Dynamics_of_universal_joints_its_failures_and_some_propositions_for_practically_improving_its_performance_and_life_expectancy}{Dynamics of universal joints}
    \end{enumerate}
\end{frame}

\begin{frame}[t]{Belt}
    \framesubtitle{Visualisation}
    \vspace{-0.5cm}
    \begin{figure}[H]
        \begin{subfigure}{0.49\textwidth}
            \centering\includegraphics[height=2.6cm,width=1\textwidth,keepaspectratio]{belt_kinematics_1.png}
            % \caption{capture1}
            \label{fig:belt_kinematics_1.png}
        \end{subfigure}
        \begin{subfigure}{0.49\textwidth}
            \centering\includegraphics[height=2.6cm,width=1\textwidth,keepaspectratio]{belt_1.png}
            % \caption{capture2}
            \label{fig:belt_1.png}
        \end{subfigure}

        \begin{subfigure}{0.49\textwidth}
            \centering\includegraphics[height=2.6cm,width=1\textwidth,keepaspectratio]{belt_kinematics_2.png}
            % \caption{capture1}
            \label{fig:belt_kinematics_2.png}
        \end{subfigure}
        \begin{subfigure}{0.49\textwidth}
            \centering\includegraphics[height=3.1cm,width=1\textwidth,keepaspectratio]{belt_2.jpg}
            % \caption{capture2}
            \label{fig:belt_2.jpg}
        \end{subfigure}
    \end{figure}
\end{frame}

\begin{frame}[t]{Belt}
    \framesubtitle{Types of belts}
    \vspace{-1.2cm}
    \begin{figure}[H]
        \centering\includegraphics[height=6cm,width=1\textwidth,keepaspectratio]{belt_types.png}
        \caption*{\textit{а}) flat (плоская), \textit{б}) vee belt (клиновидная), \textit{в}) round (круглая), \textit{г}) timing (toothed, зубчатый)}
        \label{fig:belt_types.png}
    \end{figure}

\end{frame}

\begin{frame}[t]{Belt}
    \framesubtitle{Drive kinematics}
    \begin{itemize}
        \item Linear velocity of a pulley --- $v_1=\omega_1 \frac{d_1}{2}$, $d$ --- diameter of a pulley (шкив)
        \item Length of pulley --- $l = 2a + \frac{\pi}{2}(d_1 + d_2) + \dfrac{(d_2-d_1)^2}{4a}$, where $a$ --- distance between center of pulleys.
    \end{itemize}
\end{frame}


\begin{frame}[t]{Belt}
    \framesubtitle{What can be interesting to find (queries)}
    \begin{itemize}
        \item How to find the appropriate diameter of a pulley
        \item Min and max distance between pulleys
        \item Appropriate angle of covering the pulley
    \end{itemize}
\end{frame}

\begin{frame}[t]{Belt}
    \framesubtitle{Features and facts}
    \begin{itemize}
        \item Simple design and operation, relatively low cost. 
        \item Smooth and quiet operation due to elasticity belt. 
        \item Possibility to transfer power over long distances (with V-belts up to 15 m) at speed up to 100 m/s. 
        \item Softening of vibrations and shocks due to elasticity of the belt. 
        \item Possibility to protect machines from overloading due to elastic belt tension and slippage  
        \item Reduced requirements for axle alignment 
        shafts.
    \end{itemize}
\end{frame}

\begin{frame}[t]{Belt}
    \framesubtitle{Reference material}
    \begin{enumerate}
        \item \textbf{Other names}: ременная передача
        \item \href{https://en.m.wikipedia.org/wiki/Belt_(mechanical)}{Belt drive (wiki)}
        \item \textit{"Теория механизмов и машин" Артоболевский И. И. 1988, pdf pages 166--168 }
        \item \href{https://studfile.net/preview/2156455/}{Детали машин. 9 лекция}
        \item \href{https://youtu.be/CP_b7bzM9nQ}{Belt formulas}
        \item \href{https://youtu.be/Lr5E-WrTajs}{Ременная передача видео}
    \end{enumerate}
\end{frame}


\begin{frame}[t]{Chain}
    \framesubtitle{Visualisation}
    \vspace{-1cm}
    \begin{figure}[H]
        \begin{subfigure}[c]{0.49\textwidth}
            \centering\includegraphics[height=6cm,width=1\textwidth,keepaspectratio]{belt_kinematics_1.png}
        \end{subfigure}
        \begin{subfigure}[c]{0.49\textwidth}
            \centering\includegraphics[height=6cm,width=1\textwidth,keepaspectratio]{chain_1.jpg}
            % \caption{capture2}
            \label{fig:chain_1.jpg}
        \end{subfigure}

        \begin{subfigure}[c]{0.5\textwidth}
            \centering\includegraphics[height=6cm,width=1\textwidth,keepaspectratio]{chain_2.jpg}
            % \caption{capture2}
            \label{fig:chain_2.jpg}
        \end{subfigure}
    \end{figure}
\end{frame}

\begin{frame}[t]{Chain}
    \framesubtitle{Types of chain transmissions}
    \begin{figure}[H]
        \centering\includegraphics[height=6cm,width=1\textwidth,keepaspectratio]{chain_types.png}
        % \caption{caption_name}
        \label{fig:chain_types.png}
    \end{figure}
\end{frame}

\begin{frame}[t]{Chain}
    \framesubtitle{Drive kinematics}
    Almost the same as in belt. The main difference, that max angle of sprocket covering by chain is $120^\circ$.

    Distance between centers can be fount $a = (30 - 50)P$, where $P$ --- chain pitch.
\end{frame}


\begin{frame}[t]{Chain}
    \framesubtitle{What can be interesting to find (queries)}
    \begin{itemize}
        \item Amount of tooth in sprockets.
        \item How to find a length of the chain 
    \end{itemize}
\end{frame}

\begin{frame}[t]{Chain}
    \framesubtitle{Features and facts}
    \begin{itemize}
        \item Compared to gears, chain transmissions can transmit motion between shafts at large center distance (up to 5 m)
        \item Compared to belt drives, chain transmissions are more compact, transmit more power, can be used within a considerable range of axle spacing, ensure constant transmission ratio (no slipping); 
        \item can transmit motion with one chain to several sprockets. 
        \item Irregularity of sprocket rotation.
        \item The necessity of a high accuracy of the transmission assembly.
    \end{itemize}
\end{frame}

\begin{frame}[t]{Chain}
    \framesubtitle{Reference material}
    \begin{enumerate}
        \item \textbf{Other names}: цепная передача
        \item \href{https://en.wikipedia.org/wiki/Roller_chain}{Roller chain (wiki)}
        \item \textit{"Теория механизмов и машин" Артоболевский И. И. 1988, pdf pages 166--168 }
        \item \href{https://studfile.net/preview/2156460/}{Детали машин. 10 лекция}
        \item \href{https://www.youtube.com/watch?v=F7o3LOtKEA8}{Sprockets \& Chains For Engineers}
        \item \href{https://studfile.net/preview/4421819/page:3/}{Расчет цепной передачи}
        \item \href{https://youtu.be/ps3yeekVz5I}{Цепная передача}
    \end{enumerate}
\end{frame}

\begin{frame}[t]{Geneva drive}
    \framesubtitle{Visualisation}
    \vspace{-0.5cm}
    \begin{figure}[H]
        \begin{subfigure}{0.49\textwidth}
            \centering\includegraphics[height=6cm,width=1\textwidth,keepaspectratio]{geneva_kinematics.png}
        \end{subfigure}
        \begin{subfigure}{0.49\textwidth}
            \href{https://en.wikipedia.org/wiki/Geneva_drive\#/media/File:Geneva_mechanism_6spoke_animation.gif}{
                \centering\includegraphics[height=6cm,width=1\textwidth,keepaspectratio]{Geneva_drive_video_preview.png}}
        \end{subfigure}
    \end{figure}    
\end{frame}

\begin{frame}[t]{Geneva drive}
    \framesubtitle{Example of geneva drive}
    \vspace{-0.6cm}
    \begin{figure}[H]
        \href{https://youtu.be/56l4rKLsha0}{
            \centering\includegraphics[height=6cm,width=1\textwidth,keepaspectratio]{geneva_drive_video.jpg}}
        % \caption{Click on a picture for a video}
        \label{fig:geneva_drive_video.jpg}
    \end{figure}
\end{frame}

\begin{frame}[t]{Geneva drive}
\framesubtitle{Types of geneva drive}
\vspace{-0.6cm}
    \begin{figure}[H]
        \begin{subfigure}{0.49\textwidth}
            \centering\includegraphics[height=5cm,width=1\textwidth,keepaspectratio]{geneva_inner.png}
            \caption*{Inner connection}
            \label{fig:geneva_inner.png}
        \end{subfigure}
        \begin{subfigure}{0.49\textwidth}
            \centering\includegraphics[height=5cm,width=1\textwidth,keepaspectratio]{geneva_outer.png}
            \caption*{Outer connection}
            \label{fig:geneva_outer.png}
        \end{subfigure}
    \end{figure}
\end{frame}

\begin{frame}[t]{Geneva drive}
\framesubtitle{}
    \vspace{-0.6cm}
    \begin{figure}[H]
        \centering\includegraphics[height=6cm,width=1\textwidth,keepaspectratio]{geneva_plot.png}
        \caption*{Angular velocity and acc diagram of output link}
        \label{fig:geneva_plot.png}
    \end{figure}
\end{frame}

% \begin{frame}[t]{Geneva drive}
%     \framesubtitle{What can be interesting to find (queries)}

% \end{frame}

\begin{frame}[t]{Geneva drive}
    \framesubtitle{Features and facts}
    \begin{itemize}
        \item The best application --- when you want to have a constant velocity in input link and some fancy behavior with stopping --- on output link. 
    \end{itemize}
\end{frame}

\begin{frame}[t]{Geneva drive}
    \framesubtitle{Reference material}
    \begin{enumerate}
        \item \textbf{Other names}: мальтийский крест
        \item \href{https://en.wikipedia.org/wiki/Geneva_drive}{Geneva drive (wiki)}
        \item \textit{"Теория механизмов и машин" Артоболевский И. И. 1988, pdf pages 172--174 }
        \item \href{https://www.youtube.com/watch?v=1lyWywC_z4o}{How to draw a geneva drive}
        \item \href{https://www.instructables.com/Make-Geneva-Wheels-of-Any-Size-in-a-Easier-Way/}{Make a geneva wheels of any size}
        \item \href{https://7universum.com/ru/tech/archive/item/5061}{Structural synthesis of geneva wheels (rus)}
    \end{enumerate}
\end{frame}

\begin{frame}[t]{Friction drive}
    \framesubtitle{Visualisation}
    \vspace{-0.5cm}
    \begin{figure}[H]
        \begin{subfigure}{0.49\textwidth}
            \centering\includegraphics[height=3cm,width=1\textwidth,keepaspectratio]{friction_kinematics.png}
            % \caption{capture1}
            \label{fig:friction_kinematics.png}
        \end{subfigure}
        \begin{subfigure}{0.49\textwidth}
            \centering\includegraphics[height=3cm,width=1\textwidth,keepaspectratio]{friction_1.jpg}
            % \caption{capture2}
            \label{fig:friction_1.jpg}
        \end{subfigure}
    
        \begin{subfigure}{0.49\textwidth}
            \centering\includegraphics[height=3cm,width=1\textwidth,keepaspectratio]{friction_2_kinematics.png}
            % \caption{capture3}
            \label{fig:friction_2_kinematics.png}
        \end{subfigure}
        \begin{subfigure}{0.49\textwidth}
            \centering\includegraphics[height=3cm,width=1\textwidth,keepaspectratio]{friction_2.jpg}
            % \caption{capture4}
            \label{fig:friction_2.jpg}
        \end{subfigure}
    \end{figure}
\end{frame}

\begin{frame}[t]{Friction drive}
    \framesubtitle{Continuously Variable Transmission (CVT) Video}
    \vspace{-0.6cm}
    \begin{figure}[H]
        \href{https://youtu.be/PEq5_b4LWNY}{
            \centering\includegraphics[height=6cm,width=1\textwidth,keepaspectratio]{friction_variator_preview.jpg}}
        % \caption{Click on a picture for a video}
        \label{fig:friction_variator_preview.jpg}
    \end{figure}
\end{frame}

\begin{frame}[t]{Friction drive}
    \framesubtitle{Features and facts}
    \begin{itemize}
        \item Simple design and maintenance.
        \item Smooth motion transmission and noiseless operation.
        \item Large kinematic capabilities (conversion of rotary motion into translational motion, stepless speed change, reversing on the fly, gear engagement and disengagement on the fly without stopping)
        \item Gear ratio varies due to slippage.
        \item Necessity of using specially designed shaft supports with clamping devices.
    \end{itemize}
\end{frame}

\begin{frame}[t]{Friction drive}
    \framesubtitle{Reference material}
    \begin{enumerate}
        \item \textbf{Other names}: фрикционная передача
        \item \href{https://en.m.wikipedia.org/wiki/Friction_drive}{Friction drive (wiki)}
        \item \textit{"Теория механизмов и машин" Артоболевский И. И. 1988, pdf pages 141--146 }
        \item \href{https://studfile.net/preview/2156467/page:2/}{Детали машин. 22 лекция, 2 страница}
        \item \href{https://www.youtube.com/watch?v=uCEvBGT8twM}{CVT --- how it works}
        \item \href{https://youtu.be/H1Dde7frZx8}{видео о фрикционных передача}
    \end{enumerate}
\end{frame}

\begin{frame}[t]{Gears}
    \framesubtitle{Visualisation}
    \vspace{-0.6cm}
    \begin{figure}[H]
        \begin{subfigure}[c]{0.32\textwidth}
            \centering\includegraphics[height=6cm,width=1\textwidth,keepaspectratio]{friction_kinematics.png}
            % \caption{capture1}
        \end{subfigure}
        \begin{subfigure}[c]{0.32\textwidth}
            \centering\includegraphics[height=6cm,width=1\textwidth,keepaspectratio]{gear_1.png}
            % \caption{capture2}
            \label{fig:gear_1.png}
        \end{subfigure}
        \begin{subfigure}[c]{0.32\textwidth}
            \centering\includegraphics[height=6cm,width=1\textwidth,keepaspectratio]{gear_2.png}
            % \caption{capture3}
            \label{fig:gear_2.png}
        \end{subfigure}
    \end{figure}
\end{frame}

\begin{frame}[t]{Gears}
    \framesubtitle{Types of Gears}
    \vspace{-0.6cm}
    \begin{figure}[H]
        \begin{subfigure}{0.49\textwidth}
            \centering\includegraphics[height=2.2cm,width=1\textwidth,keepaspectratio]{gear_1.png}
            \caption{Spur and helical gears}
        \end{subfigure}
        \begin{subfigure}{0.49\textwidth}
            \centering\includegraphics[height=2.2cm,width=1\textwidth,keepaspectratio]{gear_3.png}
            \caption{Rack and pinion}
            \label{fig:gear_3.png}
        \end{subfigure}
    
        \begin{subfigure}{0.49\textwidth}
            \centering\includegraphics[height=2.2cm,width=1\textwidth,keepaspectratio]{bevel_gear_1.png}
            \caption{bevel gear}
            \label{fig:bevel_gear_1.png}
        \end{subfigure}
        \begin{subfigure}{0.49\textwidth}
            \centering\includegraphics[height=2.2cm,width=1\textwidth,keepaspectratio]{gear_worm_1.png}
            \caption{worm gear}
            \label{fig:gear_worm_1.png}
        \end{subfigure}
    \end{figure}
\end{frame}

\begin{frame}[t]{Gears}
    \framesubtitle{Gear classification (Video)}
    \vspace{-0.6cm}
    \begin{figure}[H]
        \href{https://youtu.be/9n0Kn_0AHB4}{
            \centering\includegraphics[height=6cm,width=1\textwidth,keepaspectratio]{gear_classification_preview.jpg}}
        % \caption{Click on a picture for a video}
        \label{fig:gear_classification_preview.jpg}
    \end{figure}
\end{frame}

\begin{frame}[t]{Gears}
    \framesubtitle{Gear ratio calculation methods}
    \begin{enumerate}
        \item Common one, when you have a simple gear train. (Artobolevskii, pdf page 150)
        \item When you have a planetary gearset, fundamental formula of the planetary gear train (Формула Виллиса для дифференциалов). (Artobolevskii, pdf page 154 --- 166)
        \item When you have a planetary gearset, tabular method (Norton R., pdf page 550 --- 551)
    \end{enumerate}
\end{frame}

\begin{frame}[t]{Gears}
    \framesubtitle{Features and facts}
    \begin{itemize}
        \item Consistency of transmission ratio.
        \item Reliability and durability of operation.
        \item Large range of transferable speeds.
        \item High efficiency.
        \item The need for high accuracy of fabrication and assembly
    \end{itemize}
\end{frame}

\begin{frame}[t]{Gears}
    \framesubtitle{Reference material}
    \begin{enumerate}
        \item \textbf{Other names}: зубчатая передача
        \item \href{https://en.wikipedia.org/wiki/Gear}{Gears (wiki)}
        \item \textit{"Теория механизмов и машин" Артоболевский И. И. 1988, pdf pages 145--166 }
        \item \href{https://studfile.net/preview/2156468/}{Детали машин. 5-8 лекции}
        \item \textit{"Design of machinery" Robert L. Norton, pdf pages 517--557 } \textbf{2.0 --- 2.11}
        \item \href{https://youtu.be/ThVIKgBucUk}{Видео о зубчатых передачах}
    \end{enumerate}
\end{frame}

\begin{frame}[t]{Ballscrew}
    \framesubtitle{Visualisation}
    \vspace{-0.5cm}
    \begin{figure}[H]
        \begin{subfigure}{0.49\textwidth}
            \centering\includegraphics[height=6cm,width=1\textwidth,keepaspectratio]{H_sd.png}
        \end{subfigure}
        \begin{subfigure}{0.49\textwidth}
            \href{https://www.youtube.com/watch?v=oEPrXljqeHA}{
                \centering\includegraphics[height=6cm,width=1\textwidth,keepaspectratio]{ballscrew_preview.jpg}}
        \end{subfigure}
    \end{figure} 
\end{frame}

\begin{frame}[t]{Ballscrew}
    \framesubtitle{Types of ballscrew (Video)}
    \vspace{-0.6cm}
    \begin{figure}[H]
        \href{https://youtu.be/1P6nbs1-4dQ}{
            \centering\includegraphics[height=6cm,width=1\textwidth,keepaspectratio]{ballscrew_types_preview.jpg}}
        % \caption{Click on a picture for a video}
        \label{fig:ballscrew_types_preview.jpg}
    \end{figure}
\end{frame}

\begin{frame}[t]{Ballscrew}
    \framesubtitle{Drive kinematics (1)}
    \href{https://studfile.net/preview/4295396/page:16/}{Helical gear kinematics (rus)}
    
    Artobolevskii, pdf page 27
\end{frame}

\begin{frame}[t]{Ballscrew}
    \framesubtitle{Features and facts}
    \begin{itemize}
        \item Conversion of fast rotational motion of the master element into slow rectilinear motion of the slave element.
        \item Conversion of a small torque at the leading element into a significant force at the moving linear element;.
        \item Realization of the self-locking phenomenon
        \item Simplicity of design, compactness, reliability.
    \end{itemize}
\end{frame}

\begin{frame}[t]{Ballscrew}
    \framesubtitle{Reference material}
    \begin{enumerate}
        \item \textbf{Other names}: шарико-винтовая передача
        \item \href{https://en.wikipedia.org/wiki/Ball_screw}{Ball screw (wiki)}
        \item \textit{"Теория механизмов и машин" Артоболевский И. И. 1988, pdf pages 166--168 }
        \item \href{https://studfile.net/preview/2156460/}{Детали машин. 10 лекция}
        \item \href{https://youtu.be/kcrnG13NKCw}{Передача винт-гайка}
    \end{enumerate}
\end{frame}

\begin{frame}[c]{How to use provided materials}
\framesubtitle{Guidline}
    \Huge
    \centering
    Live Demo
\end{frame}


\fbckg{fibeamer/figs/last_page.png}
\frame[plain]{}

\end{document}