% \documentclass[aspectratio=169,notes]{beamer}
\documentclass[aspectratio=169]{beamer}
\usetheme[faculty=phil]{fibeamer}
\usepackage{polyglossia}
\setmainlanguage{english} %% main locale instead of `english`, you
%% can typeset the presentation in either Czech or Slovak,
%% respectively.
\setotherlanguages{russian} %% The additional keys allow
%%
%%   \begin{otherlanguage}{czech}   ... \end{otherlanguage}
%%   \begin{otherlanguage}{slovak}  ... \end{otherlanguage}
%%
%% These macros specify information about the presentation
\title[MaM]{Mechanics and Machines, Lecture 4} %% that will be typeset on the
\subtitle{Synthesis of planar mechanisms
\\ \        \\ \   
         } %% title page.
\author{Oleg Bulichev}
%% These additional packages are used within the document:
\usepackage{ragged2e}  % `\justifying` text
\usepackage{booktabs}  % Tables
\usepackage{tabularx}
\usepackage{tikz}      % Diagrams
\usetikzlibrary{calc, shapes, backgrounds}
\usetikzlibrary{decorations.pathreplacing,calligraphy,calc,graphs}
\usepackage{amsmath, amssymb}
\usepackage{url}       % `\url`s
\usepackage{listings}  % Code listings
% \usepackage{subfigure}
\usepackage{floatrow}
\usepackage{subcaption}
\usepackage{mathtools}
\usepackage{todonotes}
\usepackage{fontspec}
\usepackage{multicol}
\usepackage{pdfpages}
\usepackage{wrapfig}
\usepackage{animate}
\usepackage{booktabs}
\usepackage{multirow}
% \usepackage{graphicx}
\usepackage{colortbl}

\graphicspath{{resources/}}
\frenchspacing

\setbeamertemplate{caption}[numbered]
\usetikzlibrary{graphs}

% \usepackage[backend=biber,style=ieee,autocite=footnote]{biblatex}
% \addbibresource{biblio.bib}
% \DefineBibliographyStrings{english}{%
%   bibliography = {References},}

\newcommand{\oleg}[2][] {\todo[color=red, #1] {OLEG:\\ #2}}
\newcommand{\fbckg}[1]{\usebackgroundtemplate{\includegraphics[width=\paperwidth]{#1}}}%frame background

\usepackage[framemethod=TikZ]{mdframed}
\newcommand{\dbox}[1]{
\begin{mdframed}[roundcorner=3pt, backgroundcolor=yellow, linewidth=0]
\vspace{1mm}
{#1}
\vspace{1mm}
\end{mdframed}
}

\begin{document}
\setlength{\abovedisplayskip}{0pt}
\setlength{\belowdisplayskip}{0pt}
\setlength{\abovedisplayshortskip}{0pt}
\setlength{\belowdisplayshortskip}{0pt}

\fbckg{fibeamer/figs/title_page.png}
\frame[c]{\setcounter{framenumber}{0}
    \usebeamerfont{title}%
    \usebeamercolor[fg]{title}%
    \begin{minipage}[b][6.5\baselineskip][b]{\textwidth}%
        \textcolor{black}{\raggedright\inserttitle}
    \end{minipage}
    % \vskip-1.5\baselineskip

    \usebeamerfont{subtitle}%
    \usebeamercolor[fg]{framesubtitle}%
    \begin{minipage}[b][3\baselineskip][b]{\textwidth}
        \raggedright%
        \insertsubtitle%
    \end{minipage}
    \vskip.25\baselineskip
}
%   \frame[c]{\maketitle}

\fbckg{fibeamer/figs/common.png}

\note{\scriptsize \begin{itemize}
        \item нужно дозакодить генетику и пофиксить траекторию генерацию. А то основанная на комплексных числах не понятная. Попробовать ее расписать в обычной форме (проблема в том, что требуется знать все углы между узлами, а это очень странно с точки зрения инженерии).
        \item Рассмотреть траекторную генерацию 1ый пример (аналитика) и потом оптимизационную и сравнить.
        \item По факту эта тема сыровата.
        \item По хорошему еще раз рассчитать графический и реальные методы и сравнить (а то там явно багуля где-то)
    \end{itemize}}

\begin{frame}[c]{Difference between Analysis and Synthesis}
    \framesubtitle{}
    \vspace{-0.6cm}
    \begin{block}{Analysis}
        Analysis allows determining whether a given system will comply with certain requirements or not.

        During <<Theoretical Mechanics>> we only analyzed systems. We knew all dimensions and tried to find positions, velocities, accelerations.
    \end{block}

    \begin{block}{Synthesis}
        Synthesis is the design of a mechanism so that it complies with previously specified requirements.

        We know, that it should work on uneven terrain, and we are trying to design the robot with such possibilities. 
    \end{block}
\end{frame}

\begin{frame}[t]{Types of Synthesis}
\framesubtitle{}
\begin{columns}[T,onlytextwidth]
    \begin{column}{0.45\textwidth}
        \begin{exampleblock}{Structural}
            This synthesis deals with the \textbf{topological} and \textbf{structural study} of mechanisms. \smallskip
            
            It only considers the interconnectivity pattern of the links so that the results are unaffected by the changes in the geometric properties of the mechanisms.
        \end{exampleblock}
    \end{column}
    \begin{column}{0.45\textwidth}
        \begin{exampleblock}{Dimensional}
            It focuses on the problem of \textbf{obtaining the dimensions of a predefined mechanism} that has to comply with certain given requirements. \smallskip
            
            It will be necessary to define the dimension of the links and the position of the supports, among others.
        \end{exampleblock}
    \end{column}
\end{columns}
\end{frame}

\begin{frame}[t]{Structural Synthesis}
    \framesubtitle{2 questions, which should be answered}
    \vspace{-0.6cm}
    \begin{figure}[H]
        \begin{minipage}{0.50\textwidth}
            \footnotesize
            \textbf{1) Synthesis of type or Reuleaux synthesis:} What type of mechanism is more suitable? What type of elements will it be made of? Can it be formed by linkages, gears, flexible elements or cams?
            
            Different configurations are developed according to the pre-established requirements. The criteria to value the different characteristics of the mechanism are set.
            \label{reuleaux_synthesis.png}
        \end{minipage}\hfill
        \begin{minipage}{0.48\textwidth}
            \centering\includegraphics[height=3cm,width=1\textwidth,keepaspectratio]{reuleaux_synthesis.png}
        \end{minipage}
    \end{figure}
    \vspace{-0.6cm}
    
    \begin{figure}[H]
        \begin{minipage}{0.50\textwidth}
            \footnotesize
            \textbf{2) Synthesis of number or Grübler synthesis:} In the case of a linkage, it determines the number of links and their configuration. 
            \label{grubler_synthesis.png}
        \end{minipage}\hfill
        \begin{minipage}{0.48\textwidth}
            \centering\includegraphics[height=3cm,width=1\textwidth,keepaspectratio]{grubler_synthesis.png}
        \end{minipage}
    \end{figure}
    \end{frame}

\begin{frame}[t]{Structural Synthesis: Case Study}
    \framesubtitle{Structural synthesis problem}
    \only<1-2>{\large\begin{block}{Question}
            What the optimal number of legs should be in such robot mover?
        \end{block}}
    \only<2>{\large\begin{alertblock}{Answer}
            \centering Robot should have \textbf{8-14 legs} in total!
        \end{alertblock}}
\end{frame}

\begin{frame}[t]{Structural Synthesis: Case Study}
\framesubtitle{Criteria}
\vspace{-0.6cm}
\begin{figure}[H]
    \begin{minipage}{0.58\textwidth}
        \centering\includegraphics[height=2cm,width=1\textwidth,keepaspectratio]{f1.png}
    \end{minipage}\hfill
    \begin{minipage}{0.40\textwidth}
        More legs $\rightarrow$ higher data discretisation
        \label{fig:f1.png}
    \end{minipage}
\end{figure}
\vspace{-0.6cm}

\begin{figure}[H]
    \begin{minipage}{0.58\textwidth}
        \centering\includegraphics[height=2cm,width=1\textwidth,keepaspectratio]{f2.png}
    \end{minipage}\hfill
    \begin{minipage}{0.40\textwidth}
        More legs $\rightarrow$ longer robot $\rightarrow$ cannot pass though crooked terrains
        \label{fig:f2.png}
    \end{minipage}
\end{figure}
\vspace{-0.6cm}

\begin{figure}[H]
    \begin{minipage}{0.58\textwidth}
        \centering\includegraphics[height=2cm,width=1\textwidth,keepaspectratio]{f3.png}
    \end{minipage}\hfill
    \begin{minipage}{0.40\textwidth}
        Amount of legs nonlinearly correlates of maximal terrain passability
        \label{fig:f3.png}
    \end{minipage}
\end{figure}
\end{frame}

\begin{frame}[c]{Structural Synthesis: Case Study}
    \framesubtitle{Technological stack}
    \vspace{-0.9cm}
    \begin{figure}[H]
        \begin{subfigure}[t]{0.32\textwidth}
            \centering\includegraphics[height=4cm,width=1\textwidth,keepaspectratio]{c1_paper.png}
            \caption*{\small Generating terrain approach \\ (Robot traverse an \textbf{artificial terrain} based on \textbf{generating parameters})}
        \end{subfigure}
        \hfill
        \begin{subfigure}[t]{0.32\textwidth}
            \centering\includegraphics[height=4cm,width=1\textwidth,keepaspectratio]{gazebo_logo.png}
            \caption*{Robot simulator}
        \end{subfigure}
        \hfill
        \begin{subfigure}[t]{0.32\textwidth}
            \centering\includegraphics[height=5.5cm,width=1\textwidth,keepaspectratio]{gen_algo.jpg}
            \caption*{Genetic algorithm \\ (OpenAI-ES)}
        \end{subfigure}
        \hfill
    \end{figure}
\end{frame}

\begin{frame}[t]{Structural Synthesis: Case Study}
    \framesubtitle{Proposed solution}
    \begin{columns}[T,onlytextwidth]
        \begin{column}{0.49\textwidth}
            \begin{figure}[H]
                \centering\includegraphics[height=3cm,width=1\textwidth,keepaspectratio]{optimization_idea.png}
                \caption*{\textbf{Idea}: Minimize number of legs without losing off-road passability}
                \label{fig{optimization_idea.png}}
            \end{figure}
        \end{column}
        \begin{column}{0.49\textwidth}
            \vspace{-2cm}
            \begin{figure}[H]
                \centering
                \begin{tikzpicture}
                    % Include the image in a node
                    \node [above right, inner sep=0] (image) at (0,0)
                    {\centering\includegraphics[height=2.5cm,width=1\textwidth,keepaspectratio]{best_gen_robot.jpg}};
                    % Create scope with normalized axes
                    \begin{scope}[
                            x={($ 0.1*(image.south east)$)},
                            y={($ 0.1*(image.north west)$)}]

                        % Labels
                        \draw [green, very thick,
                            decorate,
                            decoration = {brace,
                                    raise=5pt,
                                    amplitude=5pt,
                                    aspect=0.5}] (1.4,3.6) --  (8.1,6.8)
                        node[rounded corners=3pt, pos=0.5,above left =14pt,black,fill=white]{\tiny $(\gamma - 1) h_{\text{leg}}sin(\alpha)$};

                        \draw[stealth-, very thick,green] (9.5,7.8) -- (7.8,1.94);
                        \draw[stealth-, very thick,green] (1.5,2.8) -- (7,1)
                        node[rounded corners=3pt,right,black,fill=white]{\tiny $\gamma = 6$};

                        \draw[thin,green] (6.7,4) -- (5.75,9);
                        \draw[thin,green] (4.85,3.5) -- (5.75,9);
                        \draw[thin,green,stealth-stealth] (6.32,6) arc (-79.2:-99.2:3) node [rounded corners=3pt,below = 2pt,black,fill=white, midway] {\tiny $\alpha$};
                        % \draw[very thick,green] (8,6) -- (5,8);
                        % \draw[very thick,green] (0.5,2.5) rectangle (4.2,9)
                        % node[below left,black,fill=green]{\small test};
                    \end{scope}
                \end{tikzpicture}
                % \caption*{}
                \label{fig:best_gen_robot.jpg}
            \end{figure}
            \vspace{-1cm}
            {\footnotesize
                \begin{eqnarray*}
                    % \resizebox{0.9\hsize}{!}{
                    F \rightarrow max = \beta \left( {\omega}_{1} \cdot \overbrace{\delta}^{\text{Distance}} + {\omega}_{2} \cdot \overbrace{\frac{1}{(\gamma - 1) h_{\text{leg}}sin(\alpha)}}^{\text{Simplified body length}}\right) +\\ \nonumber + (1 - \beta) {\delta}^{{\omega}_{1}} {\left( \frac{1}{(\gamma - 1)h_{\text{leg}}sin(\alpha)}\right)}^{{\omega}_{2}}
                    % }
                \end{eqnarray*}
            }
            % \vspace{1pt}

            $\beta$ is adaptive parameter, \\ ${\omega}_{1,2} \in  [ 0..1 ] $ are the weight coefficients.
        \end{column}
    \end{columns}
\end{frame}

\begin{frame}[t]{Structural Synthesis: Case Study}
    \framesubtitle{Video: The story of one generated robot}
    \vspace{-0.6cm}
    \begin{figure}[H]
        % \href{run:./videos/pass_rand_terr.mp4}{
        \href{https://youtu.be/DcovvkTZgsg}{
            \centering\includegraphics[height=6cm,width=1\textwidth,keepaspectratio]{genetic_video_preview.jpg}}
        % \caption{Click on a picture for a video}
    \end{figure}
\end{frame}

\begin{frame}[t]{Structural Synthesis: Case Study}
    \framesubtitle{Particular results: $\omega_1 = 0.6$, $\omega_2 = 0.4$}
    \vspace{-0.6cm}

    \begin{table}[H]
        \centering
        \begin{tabular}{c|c|c|c|c}
         & \textbf{\begin{tabular}[c]{@{}c@{}}Terrain\\ types\end{tabular}} & \textbf{No. Legs} & \textbf{\begin{tabular}[c]{@{}c@{}}Angle b/w\\ neighbor legs\end{tabular}} & \textbf{No. individuals} \\
         \hline
         \rule{0cm}{0.5cm}
        \textbf{Phase 1} &  & \cellcolor[HTML]{DAE8FC}12 & 73 & 200 \\ \cline{1-1} \cline{3-5} 
         & \multirow{-2}{*}{\begin{minipage}{2.5cm}\includegraphics[height=3cm,width=2.5cm,keepaspectratio]{terrain_1.jpg}\end{minipage}} & \cellcolor[HTML]{DAE8FC}12 & 72 &  \\ [0.5cm] \cline{3-4} 
         & \begin{minipage}{2.5cm}\includegraphics[height=3cm,width=2.5cm,keepaspectratio]{terrain_2.jpg}\end{minipage} & \cellcolor[HTML]{DAE8FC}10 & 68 &  \\ [0.5cm] \cline{3-4}
        \multirow{-3}{*}{\textbf{Phase 2}} & \begin{minipage}{2.5cm}\includegraphics[height=3cm,width=2.5cm,keepaspectratio]{terrain_3.jpg}\end{minipage} & \cellcolor[HTML]{DAE8FC}12 & 77 & \multirow{-3}{*}{55}
        \end{tabular}
        % \caption*{\large\centering\textbf{Summary}: created robot should have 10-12 legs in total}
        \end{table}

\end{frame}

\begin{frame}[t]{Structural Synthesis: Case Study}
    \framesubtitle{Global results}
    \begin{columns}[T,onlytextwidth]
        \begin{column}{0.49\textwidth}
            Based on fitness function the number of legs range starts from 8 till 14 for different $\omega$ values. 
            
            It can be explained by static stability criteria. In such case 4 legs will touch the ground.    
        \end{column}
        \begin{column}{0.49\textwidth}
            \vspace{-1.5cm}
            \begin{figure}[H]
                \centering\includegraphics[height=5cm,width=1\textwidth,keepaspectratio]{box_plot_structural_synthesis.png}
                \caption*{Correlation between amount of legs and passed distance by best robot individuals from several experiments}
                \label{fig:box_plot_structural_synthesis.png}
            \end{figure}
        \end{column}
    \end{columns}
\end{frame}


\begin{frame}[t]{Dimensional Synthesis}
\framesubtitle{Functional Generation}
Pre-established conditions refer to the relation between the input and output motions. These are defined by variables $\phi$ and $\psi$, that indentify their positions.
    \begin{figure}[H]
        \centering\includegraphics[height=3cm,width=1\textwidth,keepaspectratio]{func_gen1.png}
        % \caption{caption_name}
        \label{fig:func_gen1.png}
    \end{figure}
\end{frame}

\begin{frame}[t]{Functional Generation}
\framesubtitle{Case Study}
    \begin{columns}[T,onlytextwidth]
        \begin{column}{0.49\textwidth}
            Function generation can be used to design mechanisms that carry out mathematical operations: addition, differentiation, integration or a combination of them. \smallskip

            The first computers were mechanical devices based on this type of mechanisms.
        \end{column}
        \begin{column}{0.49\textwidth}
            \vspace{-1cm}
            \begin{figure}[H]
                \centering\includegraphics[height=6cm,width=1\textwidth,keepaspectratio]{func_gen2.png}
                \label{fig:func_gen2.png}
            \end{figure}
        \end{column}
    \end{columns}
\end{frame}

\begin{frame}[t]{Dimensional Synthesis}
\framesubtitle{Trajectory Generation}
    \begin{columns}[T,onlytextwidth]
        \begin{column}{0.49\textwidth}
            It studies and provides methods in order to obtain mechanisms in which one of the points describes a given trajectory
        \end{column}
        \begin{column}{0.49\textwidth}
            \vspace{-1cm}
            \begin{figure}[H]
                \centering\includegraphics[height=6cm,width=1\textwidth,keepaspectratio]{traj_gen1.png}
                % \caption{caption_name}
                \label{fig:traj_gen1.png}
            \end{figure}
        \end{column}
    \end{columns}
\end{frame}

\begin{frame}[t]{Trajectory Generation}
    \framesubtitle{Case Study: Video}
    \vspace{-0.6cm}
    \begin{figure}[H]
        \href{https://youtu.be/DfznnKUwywQ}{
            \centering\includegraphics[height=6cm,width=1\textwidth,keepaspectratio]{disney_research_preview.jpg}}
        % \caption{Click on a picture for a video}
        \label{fig:disney_research_preview.jpg}
    \end{figure}
\end{frame}

\begin{frame}[t]{Types of solving methods}
\framesubtitle{}
    \begin{exampleblock}{Graphical}
        These methods are very didactic and help us to understand the problem in an easy way. However, they offer a limited range of possibilities.
    \end{exampleblock}
    
    \begin{exampleblock}{Analytical}
        They solve the problem by means of mathematical equations based on the requirements.
    \end{exampleblock}
    
    \begin{exampleblock}{Optimization-technique-based}
        They can find the optimal solution to the problem by means of the minimization of an objective function and the establishment of a series of restrictions. Different optimization techniques can be used.
    \end{exampleblock}

\end{frame}

\begin{frame}[t]{Function Generation Synthesis}
\framesubtitle{How to represent data}

\vspace{-0.6cm}
\begin{figure}[H]
    \begin{minipage}{0.58\textwidth}
        \centering\includegraphics[height=2.8cm,width=1\textwidth,keepaspectratio]{func_gen3.png}
    \end{minipage}\hfill
    \begin{minipage}{0.40\textwidth}
        1) Continuous function (desired and generated)
        \label{fig:func_gen3.png}
    \end{minipage}
\end{figure}
\vspace{-0.6cm}

\begin{figure}[H]
    \begin{minipage}{0.58\textwidth}
        \centering\includegraphics[height=2cm,width=1\textwidth,keepaspectratio]{func_gen4.png}
    \end{minipage}\hfill
    \begin{minipage}{0.40\textwidth}
        2) Precision points
        \label{fig:func_gen4.png}
    \end{minipage}
\end{figure}
\vspace{-0.6cm}

\begin{figure}[H]
    \begin{minipage}{0.58\textwidth}
        \centering\includegraphics[height=2cm,width=1\textwidth,keepaspectratio]{func_gen5.png}
    \end{minipage}\hfill
    \begin{minipage}{0.40\textwidth}
        \textit{It's representation for optimation based algorithms}. Error function (difference between generated and desired)
        \label{fig:func_gen5.png}
    \end{minipage}
\end{figure}
\end{frame}

\begin{frame}[t]{Function Generation Synthesis}
\framesubtitle{Graphical Method (1)}
    \begin{columns}[T,onlytextwidth]
        \begin{column}{0.39\textwidth}
            As an example, we will generate a four-bar mechanism in which a rotated angle of the input link between positions $1$ and $2$, $\phi_12$, corresponds to a rotated angle of output link $\psi_12$
        \end{column}
        \begin{column}{0.59\textwidth}
            \vspace{-0.6cm}
            \begin{figure}[H]
                \centering\includegraphics[height=6cm,width=1\textwidth,keepaspectratio]{func_gen6.png}
                % \caption{caption_name}
                \label{fig:func_gen6.png}
            \end{figure}
        \end{column}
    \end{columns}
\end{frame}

\begin{frame}[t]{Function Generation Synthesis}
    \framesubtitle{Graphical Method (2): Algorithm}
    \vspace{-0.8cm}
    \begin{figure}[H]
        \centering\includegraphics[height=6cm,width=1\textwidth,keepaspectratio]{func_gen7.png}
        % \caption{caption_name}
        \label{fig:func_gen7.png}
    \end{figure}
\end{frame}

\begin{frame}[t]{Function Generation Synthesis}
    \framesubtitle{Graphical Method (3): Particular Case}
    \vspace{-0.8cm}
    \begin{figure}[H]
        \centering\includegraphics[height=6cm,width=1\textwidth,keepaspectratio]{func_gen8.png}
        % \caption{caption_name}
        \label{fig:func_gen8.png}
    \end{figure}
\end{frame}

\begin{frame}[t]{Function Generation Synthesis}
    \framesubtitle{Analytical Method (1): Freudenstein's method}
    % \vspace{-0.8cm}
    Find all lengths of four link bar mechanism if we know 3 related positions of input and output links.
    \begin{figure}[H]
        \centering\includegraphics[height=4cm,width=1\textwidth,keepaspectratio]{func_gen9.png}
        % \caption{caption_name}
        \label{fig:func_gen9.png}
    \end{figure}
\end{frame}

\begin{frame}[t]{Function Generation Synthesis}
    \framesubtitle{Analytical Method (2): full solution in pdf page 11}
    \vspace{-0.8cm}
    \begin{gather}
        \vec{a} + \vec{c} = \vec{d} + \vec{b} \\ 
        \left\{\begin{matrix*}[l]
        a\cos(\phi) + c\cos(\theta) = d + b \cos(\psi)\\ 
        a\sin(\phi)+c\sin(\theta)=b\sin(\psi)
        \end{matrix*}\right. \\ 
        \text{In order to find the relation $\psi = f(\phi)$} \\ 
        c^2 = d^2 + b^2 + a^2 + 2bd\cos(\psi) - 2ab\cos(\psi - \phi) - 2da\cos(\phi) \\ 
        R_1\cos(\psi) - R_2\cos(\phi) + R_3 = \cos(\psi - \phi), \text{ where: } \\
        \left\{\begin{matrix*}[l]
        R_1=\dfrac{d}{a};\ R_2 = \dfrac{d}{b}\\ 
        R_3 = \dfrac{d^2+b^2+a^2-c^2}{2ab}
        \end{matrix*}\right.
    \end{gather}
\end{frame}

\begin{frame}[t]{Function Generation Synthesis}
    \framesubtitle{Analytical Method (3)}
    \vspace{-0.7cm}
    \begin{figure}[H]
        \centering\includegraphics[height=6cm,width=1\textwidth,keepaspectratio]{func_gen10.png}
        % \caption{caption_name}
        \label{fig:func_gen10.png}
    \end{figure}
\end{frame}

\begin{frame}[t]{Function Generation Synthesis}
    \framesubtitle{Analytical Method (4): Comparison with graphical}
    \vspace{-0.7cm}
    \begin{figure}[H]
        \centering\includegraphics[height=6cm,width=1\textwidth,keepaspectratio]{func_gen11.png}
        % \caption{caption_name}
        \label{fig:func_gen11.png}
    \end{figure}
\end{frame}

\begin{frame}[t]{Trajectory Generation Synthesis}
\framesubtitle{Formal definition}
\vspace{-0.5cm}

Relationship between the trajectory described by a point in a link and the motion of another link, usually the input one.
\vspace{-0.5cm}

\begin{figure}[H]
    \href{https://www.geogebra.org/m/SF2rQXEp}{
        \centering\includegraphics[height=5cm,width=1\textwidth,keepaspectratio]{traj_gen2.png}}
    \label{fig:traj_gen2.png}
\end{figure}
\end{frame}

\begin{frame}[t]{Trajectory Generation Synthesis}
    \framesubtitle{Graphical Method (1)}
    \vspace{-0.5cm}
    \begin{columns}[T,onlytextwidth]
        \begin{column}{0.49\textwidth}
            This method allows finding a four-bar mechanism in which the coupler link passes through the three specified positions. The steps to follow are the next:
ones:
        \end{column}
        \begin{column}{0.49\textwidth}
            \begin{figure}[H]
                \centering\includegraphics[height=6cm,width=1\textwidth,keepaspectratio]{traj_gen3.png}
                \label{fig:traj_gen3.png}
            \end{figure}
        \end{column}
    \end{columns}
\end{frame}

\begin{frame}[t]{Trajectory Generation Synthesis}
    \framesubtitle{Graphical Method (2): Algorithm}
    \vspace{-0.7cm}
    \begin{figure}[H]
        \centering\includegraphics[height=6cm,width=1\textwidth,keepaspectratio]{traj_gen4.png}
        % \caption{caption_name}
        \label{fig:traj_gen4.png}
    \end{figure}
\end{frame}

\begin{frame}[t]{Trajectory Generation Synthesis}
    \framesubtitle{Analytical Method (1): Based on complex numbers}
    \vspace{-0.7cm}
    \begin{figure}[H]
        \centering\includegraphics[height=6cm,width=1\textwidth,keepaspectratio]{traj_gen5.png}
        \vspace{-0.7cm}

        \caption*{Full solution is \textit{on pdf page 16}}
        \label{fig:traj_gen5.png}
    \end{figure}
\end{frame}

\begin{frame}[t]{Optimal Synthesis}
    \framesubtitle{Formal definition (1)}
    \vspace{-0.7cm}
    \begin{figure}[H]
        \centering\includegraphics[height=6cm,width=1\textwidth,keepaspectratio]{opti_gen1.png}
        % \caption{caption_name}
        \label{fig:opti_gen1.png}
    \end{figure}
\end{frame}

\begin{frame}[t]{Optimal Synthesis}
    \framesubtitle{Formal definition (2)}
    \vspace{-0.7cm}
    \begin{figure}[H]
        \centering\includegraphics[height=6cm,width=1\textwidth,keepaspectratio]{opti_gen2.png}
        % \caption{caption_name}
        \label{fig:opti_gen2.png}
    \end{figure}
\end{frame}

\begin{frame}[t]{Optimal Synthesis}
    \framesubtitle{Formal definition (3)}
    \vspace{-0.7cm}
    \begin{figure}[H]
        \centering\includegraphics[height=6cm,width=1\textwidth,keepaspectratio]{opti_gen2_5.png}
        % \caption{caption_name}
        \label{fig:opti_gen2_5.png}
    \end{figure}
\end{frame}

\begin{frame}[t]{Optimal Synthesis}
    \framesubtitle{Function Generation: objective function}
    \vspace{-0.7cm}
    \begin{figure}[H]
        \centering\includegraphics[height=6cm,width=1\textwidth,keepaspectratio]{opti_gen3.png}
        % \caption{caption_name}
        \label{fig:opti_gen3.png}
    \end{figure}
\end{frame}

\begin{frame}[t]{Optimal Synthesis}
    \framesubtitle{Trajectory Generation: objective function}
        \centering \LARGE
        Starting \textit{pdf page 23}
    \end{frame}

\begin{frame}[t]{Optimal Synthesis}
\framesubtitle{Evolutionary Algorithms}
    \centering \LARGE
    Starting \textit{pdf page 28}
\end{frame}

\begin{frame}[t]{Reference material}
    \framesubtitle{}
    \begin{enumerate}
        \item \href{https://disk.yandex.ru/i/GCbdbYRq94vbxA}{Synthesis of planar mechanisms (chapter from book)}
        \item \href{https://link.springer.com/book/10.1007/978-3-319-31970-4\#toc}{\textit{"Fundamentals of Machine Theory and Mechanisms" book}}
        \item \href{https://www.geogebra.org/m/SF2rQXEp}{Collection of applets on the Synthesis of Mechanisms (Geogebra)}
        \item \href{https://www.youtube.com/playlist?list=PLH1r3LGlktdvKBfxlcIji4Shf_RrWw8on}{Mechanics of Machinery (MOM) Module 6 Synthesis of Mechanisms (YouTube)}
    \end{enumerate}
\end{frame}



\fbckg{fibeamer/figs/last_page.png}
\frame[plain]{}

\end{document}