\documentclass[12pt]{article}

\usepackage{geometry}
 \geometry{
 a4paper,
 total={170mm,257mm},
 left=20mm,
 right=10mm,
 top=10mm,
 bottom=20mm,
 headheight=75pt
 }
% \usepackage[utf8x]{inputenc}
\usepackage{fontspec}
\setmainfont{Open Sans}
\setsansfont{Noto Sans}
\usepackage{graphicx}
\usepackage{forloop}
\usepackage{subcaption}
\usepackage{url}       % `\url`s
\usepackage{floatrow}
\usepackage{hyperref}
\usepackage{cleveref}
\graphicspath{{resources_quiz_1/}}

\usepackage{fancyhdr}
\pagestyle{fancy}


\renewcommand{\headrulewidth}{0pt}
% \fancyhead[C]{}
% \fancyfoot[]{}


\newcommand\pic[1]{(\cref{#1})} %Где нужно сослаться на рисунок

\hypersetup{
    colorlinks=true,
    linkcolor=blue,
    urlcolor=cyan,
    }

% The preamble ends with the command \begin{document}
\begin{document}
\begin{center}
    \LARGE <<Introduction to Mechanical Engineering>> \\ \textbf{Midterm}
\end{center}

\textbf{Task 1}
\begin{enumerate}
    \item What does it mean? You should explain each part of this notation \pic{fig:task11}.
    \item Using which 4 basic operations you can design almost any solid part in CAD. Explain your choice with an example.
\end{enumerate}
\begin{figure}[H]
    \begin{subfigure}{0.49\textwidth}
        \centering\includegraphics[height=3cm,width=1\textwidth,keepaspectratio]{resources_quiz_1/quiz1_task1.png}
        \caption{1st subtask}
        \label{fig:resources_quiz_1/quiz1_task1.png}
    \end{subfigure}
    \begin{subfigure}{0.49\textwidth}
        \centering\includegraphics[height=3cm,width=1\textwidth,keepaspectratio]{resources_quiz_1/quiz1_task12.png}
        \caption{2nd subtask}
        \label{fig:resources_quiz_1/quiz1_task12.png}
    \end{subfigure}
\caption{Tasks 1.1}
\label{fig:task11}
\end{figure}


\textbf{Task 2}
\begin{enumerate}
    \item What the difference between lower and higher kinematic pairs and. Provide examples of both types, using kinematic scheme notation.
    \item Draw a kinematic scheme of the mechanism \pic{fig:resources_quiz_1/quiz1_task2.png}.
\end{enumerate}
\begin{figure}[H]
    \centering\includegraphics[height=5cm,width=1\textwidth,keepaspectratio]{resources_quiz_1/quiz1_task2.png}
    \caption{Task 2.2}
    \label{fig:resources_quiz_1/quiz1_task2.png}
\end{figure}

\textbf{Task 3}
\begin{enumerate}
    \item Provide at least 4 types of drives. Prof and cons.
\end{enumerate}

% \textbf{Task 4}
% \begin{enumerate}
%     \item Why do we need bearings?
%     \item How to fix radial bearing on a shaft. At least 2 possible ways.
%     \item Locating and floating bearings. What the idea besides it?
% \end{enumerate}

\textbf{Task 4}
\begin{enumerate}
    \item Could you name all highlighted parts from the picture \pic{fig:resources_quiz_1/quiz1_task5.png}?
    \item What the difference between bolden and direct extruders.
    \item Could you write the printing process, starting that you have \textit{ideal} CAD model in <<step>> format.  
\end{enumerate}
\begin{figure}[H]
    \centering\includegraphics[height=8cm,width=1\textwidth,keepaspectratio]{resources_quiz_1/quiz1_task5.png}
    \caption{Task 5.1}
    \label{fig:resources_quiz_1/quiz1_task5.png}
\end{figure}

\textbf{Task 5}
\begin{enumerate}
    \item What does stress and strain mean? Stress-strain curve. What the idea besides it? Draw some curve for ductile and brittle material. How can we modify a curve behavior for some particular material?
    \item Why do we need alloying elements? Could you provide at least 1 example?
    % \item Iron-Carbon Phase Diagram \pic{fig:resources_quiz_1/quiz1_task6.png}. What can you understand from the diagram?
\end{enumerate}
% \begin{figure}[H]
%     \centering\includegraphics[height=8cm,width=1\textwidth,keepaspectratio]{resources_quiz_1/quiz1_task6.png}
%     \caption{Task 6.3}
%     \label{fig:resources_quiz_1/quiz1_task6.png}
% \end{figure}

\textbf{Task 6}
\begin{enumerate}
    \item What types of synthesis do we have? What the main difference between them?
    \item Propose the problem of structural synthesis and the steps what should you do for solving it?
    \item What the difference between function generation and trajectory generation synthesis problems?
\end{enumerate}

\textbf{Task 7}
\begin{enumerate}
    \item You are a cleaning robot developer. You know proposed characteristics of robot inertia, and you should choose the motors for it. Robot kinematics is 2 wheel robot, with 2 supports.

    Explain your steps for choosing the motor.
\end{enumerate}

\textbf{Task 8}
\begin{enumerate}
    \item What the difference between rolling friction and sliding friction? 
\end{enumerate}
\end{document}