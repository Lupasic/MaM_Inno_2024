% \documentclass[aspectratio=169,notes]{beamer}
\documentclass[aspectratio=169]{beamer}
\usetheme[faculty=phil]{fibeamer}
\usepackage{polyglossia}
\setmainlanguage{english} %% main locale instead of `english`, you
%% can typeset the presentation in either Czech or Slovak,
%% respectively.
\setotherlanguages{russian} %% The additional keys allow
%%
%%   \begin{otherlanguage}{czech}   ... \end{otherlanguage}
%%   \begin{otherlanguage}{slovak}  ... \end{otherlanguage}
%%
%% These macros specify information about the presentation
\title[MaM]{Mechanics and Machines, Lecture 2} %% that will be typeset on the
\subtitle{Types of drives: kinematics, where to find other info
\\ Drives: friction, belts, chains, gears, universal, geneva       \\ \  
         } %% title page.
\author{Oleg Bulichev}
%% These additional packages are used within the document:
\usepackage{ragged2e}  % `\justifying` text
\usepackage{booktabs}  % Tables
\usepackage{tabularx}
\usepackage{tikz}      % Diagrams
\usetikzlibrary{calc, shapes, backgrounds}
\usepackage{amsmath, amssymb}
\usepackage{url}       % `\url`s
\usepackage{listings}  % Code listings
% \usepackage{subfigure}
\usepackage{floatrow}
\usepackage{subcaption}
\usepackage{mathtools}
\usepackage{todonotes}
\usepackage{fontspec}
\usepackage{multicol}
\usepackage{pdfpages}
\usepackage{wrapfig}
\usepackage{animate}
\usepackage{booktabs}
\usepackage{multirow}
% \usepackage{graphicx}
\usepackage{colortbl}

\graphicspath{{resources/}}
\frenchspacing

\setbeamertemplate{caption}[numbered]
\usetikzlibrary{graphs}

% \usepackage[backend=biber,style=ieee,autocite=footnote]{biblatex}
% \addbibresource{biblio.bib}
% \DefineBibliographyStrings{english}{%
%   bibliography = {References},}

\newcommand{\oleg}[2][] {\todo[color=red, #1] {OLEG:\\ #2}}
\newcommand{\fbckg}[1]{\usebackgroundtemplate{\includegraphics[width=\paperwidth]{#1}}}%frame background

\usepackage[framemethod=TikZ]{mdframed}
\newcommand{\dbox}[1]{
\begin{mdframed}[roundcorner=3pt, backgroundcolor=yellow, linewidth=0]
\vspace{1mm}
{#1}
\vspace{1mm}
\end{mdframed}
}

\begin{document}
\setlength{\abovedisplayskip}{0pt}
\setlength{\belowdisplayskip}{0pt}
\setlength{\abovedisplayshortskip}{0pt}
\setlength{\belowdisplayshortskip}{0pt}

\fbckg{fibeamer/figs/title_page.png}
\frame[c]{\setcounter{framenumber}{0}
    \usebeamerfont{title}%
    \usebeamercolor[fg]{title}%
    \begin{minipage}[b][6.5\baselineskip][b]{\textwidth}%
        \textcolor{black}{\raggedright\inserttitle}
    \end{minipage}
    % \vskip-1.5\baselineskip

    \usebeamerfont{subtitle}%
    \usebeamercolor[fg]{framesubtitle}%
    \begin{minipage}[b][3\baselineskip][b]{\textwidth}
        \raggedright%
        \insertsubtitle%
    \end{minipage}
    \vskip.25\baselineskip
}
%   \frame[c]{\maketitle}

\fbckg{fibeamer/figs/common.png}

\note{\scriptsize \begin{itemize}
        \item \
    \end{itemize}}

\begin{frame}[c]{Goal of the lecture}
    \framesubtitle{}
    \LARGE
    \centering
    Make an overview of typical drives. \\ Give a hint how to work with it. \\ Explain how to find information about particular drive.
\end{frame}

\begin{frame}[t]{Universal Joint}
    \framesubtitle{Visualisation}
    \vspace{-0.5cm}
    \begin{figure}[H]
        \begin{subfigure}{0.49\textwidth}
            \centering\includegraphics[height=6cm,width=1\textwidth,keepaspectratio]{universal_kinematics.png}
        \end{subfigure}
        \begin{subfigure}{0.49\textwidth}
            \href{https://en.wikipedia.org/wiki/Universal_joint\#/media/File:Universal_joint.gif}{
                \centering\includegraphics[height=6cm,width=1\textwidth,keepaspectratio]{cardan_video_preview.png}}
        \end{subfigure}
    \end{figure}
\end{frame}


\begin{frame}[t]{Universal Joint}
    \framesubtitle{Types of universal joint}
    \begin{figure}[H]
        \begin{subfigure}{0.32\textwidth}
            \centering\includegraphics[height=6cm,width=1\textwidth,keepaspectratio]{universal_1.png}
            \caption*{Cardan}
        \end{subfigure}
        \begin{subfigure}{0.32\textwidth}
            \href{https://en.wikipedia.org/wiki/Constant-velocity_joint\#/media/File:Double_Cardan_Joint_(animated).gif}{
                \centering\includegraphics[height=6cm,width=1\textwidth,keepaspectratio]{universal_2_video_preview.png}}
            \caption*{Double cardan joint}
        \end{subfigure}
        \begin{subfigure}{0.32\textwidth}
            \href{https://gifyu.com/image/SqMnR}{
                \centering\includegraphics[height=6cm,width=1\textwidth,keepaspectratio]{shrus_video_preview.png}}
            \caption*{Constant-velocity universal ball joint}
        \end{subfigure}
    \end{figure}
\end{frame}

\begin{frame}[t]{Universal Joint}
    \framesubtitle{Drive kinematics (1)}
    \begin{columns}[T,onlytextwidth]
        \begin{column}{0.59\textwidth}
            Angle relationship --- $\tan(\psi)=\tan(\psi')\cos(\beta)$ \\
            Angular velocities relationship --- $ \omega \cos(\beta) = \omega'(\sin^2(\psi) + \cos^2(\psi)\cos^2(\beta))$
        \end{column}
        \begin{column}{0.39\textwidth}
            \vspace{-0.9cm}
            \begin{figure}[H]
                \centering\includegraphics[height=6cm,width=1\textwidth,keepaspectratio]{universal_kinematics.png}
                % \caption{caption_name}
                \label{fig:universal_kinematics.png}
            \end{figure}
        \end{column}
    \end{columns}

\end{frame}

\begin{frame}[t]{Universal Joint}
    \framesubtitle{Features and facts}
    \begin{itemize}
        \item It's effective tool for transferring a torque for max 30 degrees.
        \item Constant-velocity universal ball joint (шрус) is not a small device and it's not easy to find it (it can be found as a car detail).
    \end{itemize}
\end{frame}

\begin{frame}[t]{Universal Joint}
    \framesubtitle{What can be interesting to find (queries)}
    \begin{enumerate}
        \item Correlation between velocities and angle between links in Universal joint
        \item Cardan dynamics
    \end{enumerate}
\end{frame}

\begin{frame}[t]{Universal Joint}
    \framesubtitle{Reference material}
    \begin{enumerate}
        \item \textbf{Other names}: cardan joint, Hooke's joint, кардан, универсальный шарнир
        \item \href{https://en.wikipedia.org/wiki/Universal_joint}{Universal joint (wiki)}
        \item \textit{"Теория механизмов и машин" Артоболевский И. И. 1988, pdf pages 168--172 }
        \item \href{https://elar.urfu.ru/bitstream/10995/102516/1/2-s2.0-85107367228.pdf}{Find U-joint parameters using quaternions}
        \item \href{https://www.researchgate.net/publication/257774799_Dynamics_of_universal_joints_its_failures_and_some_propositions_for_practically_improving_its_performance_and_life_expectancy}{Dynamics of universal joints}
    \end{enumerate}
\end{frame}

\begin{frame}[t]{Belt}
    \framesubtitle{Visualisation}
    \vspace{-0.5cm}
    \begin{figure}[H]
        \begin{subfigure}{0.49\textwidth}
            \centering\includegraphics[height=2.6cm,width=1\textwidth,keepaspectratio]{belt_kinematics_1.png}
            % \caption{capture1}
            \label{fig:belt_kinematics_1.png}
        \end{subfigure}
        \begin{subfigure}{0.49\textwidth}
            \centering\includegraphics[height=2.6cm,width=1\textwidth,keepaspectratio]{belt_1.png}
            % \caption{capture2}
            \label{fig:belt_1.png}
        \end{subfigure}

        \begin{subfigure}{0.49\textwidth}
            \centering\includegraphics[height=2.6cm,width=1\textwidth,keepaspectratio]{belt_kinematics_2.png}
            % \caption{capture1}
            \label{fig:belt_kinematics_2.png}
        \end{subfigure}
        \begin{subfigure}{0.49\textwidth}
            \centering\includegraphics[height=3.1cm,width=1\textwidth,keepaspectratio]{belt_2.jpg}
            % \caption{capture2}
            \label{fig:belt_2.jpg}
        \end{subfigure}
    \end{figure}
\end{frame}

\begin{frame}[t]{Belt}
    \framesubtitle{Types of belt transmission}

\end{frame}

\begin{frame}[t]{Belt}
    \framesubtitle{Drive kinematics (1)}

\end{frame}

\begin{frame}[t]{Belt}
    \framesubtitle{Drive kinematics (2)}

\end{frame}

\begin{frame}[t]{Belt}
    \framesubtitle{What can be interesting to find (queries)}

\end{frame}

\begin{frame}[t]{Belt}
    \framesubtitle{Reference material}

\end{frame}

\begin{frame}[t]{Chain}
    \framesubtitle{Visualisation}

\end{frame}

\begin{frame}[t]{Chain}
    \framesubtitle{Types of chain transmissions}

\end{frame}

\begin{frame}[t]{Chain}
    \framesubtitle{Drive kinematics (1)}

\end{frame}

\begin{frame}[t]{Chain}
    \framesubtitle{Drive kinematics (2)}

\end{frame}

\begin{frame}[t]{Chain}
    \framesubtitle{What can be interesting to find (queries)}

\end{frame}

\begin{frame}[t]{Chain}
    \framesubtitle{Reference material}

\end{frame}

\begin{frame}[t]{Geneva drive}
    \framesubtitle{Visualisation}

\end{frame}

\begin{frame}[t]{Geneva drive}
    \framesubtitle{Types of geneva drive}

\end{frame}

\begin{frame}[t]{Geneva drive}
    \framesubtitle{Drive kinematics (1)}

\end{frame}

\begin{frame}[t]{Geneva drive}
    \framesubtitle{Drive kinematics (2)}

\end{frame}

\begin{frame}[t]{Geneva drive}
    \framesubtitle{What can be interesting to find (queries)}

\end{frame}

\begin{frame}[t]{Geneva drive}
    \framesubtitle{Reference material}

\end{frame}

\begin{frame}[t]{Friction drive}
    \framesubtitle{Visualisation}

\end{frame}

\begin{frame}[t]{Friction drive}
    \framesubtitle{Types of friction drive}

\end{frame}

\begin{frame}[t]{Friction drive}
    \framesubtitle{Drive kinematics (1)}

\end{frame}

\begin{frame}[t]{Friction drive}
    \framesubtitle{Drive kinematics (2)}

\end{frame}

\begin{frame}[t]{Friction drive}
    \framesubtitle{What can be interesting to find (queries)}

\end{frame}

\begin{frame}[t]{Friction drive}
    \framesubtitle{Reference material}

\end{frame}

\begin{frame}[t]{Gears}
    \framesubtitle{Visualisation}

\end{frame}

\begin{frame}[t]{Gears}
    \framesubtitle{Types of Gears}

\end{frame}

\begin{frame}[t]{Gears}
    \framesubtitle{Drive kinematics (1)}

\end{frame}

\begin{frame}[t]{Gears}
    \framesubtitle{Drive kinematics (2)}

\end{frame}

\begin{frame}[t]{Gears}
    \framesubtitle{What can be interesting to find (queries)}

\end{frame}

\begin{frame}[t]{Gears}
    \framesubtitle{Reference material}

\end{frame}



\begin{frame}[t]{Reference material}
    % \Large
    \begin{itemize}
        \item \textit{"Mechanisms and Machines: Kinematics, Dynamics, and Synthesis" Michael M. Stanisic, pdf pages 21--56 } \textbf{1.1 --- 1.6}
        \item \textit{"Theory of Machines and Mechanisms" John J. Uicker, pdf pages 33--59 } \textbf{1.4 --- 1.7}
        \item \textit{"Design of machinery" Robert L. Norton, pdf pages 57--79 } \textbf{2.0 --- 2.11}
        \item \textit{"Механика. Теория механизмов и машин" Конищева О. В., pdf pages 7--23 } \\ Структурный анализ и классификация плоских механизмов
        \item \textit{"Теория механизмов и машин" Артоболевский И. И. 1988, pdf pages 21--63 } \\ Структурный анализ и классификация механизмов
              % \item \href{https://onlinemschool.com/math/library/vector/cos/}{Direction cosines (OnlineMSchool)}
    \end{itemize}
\end{frame}

\fbckg{fibeamer/figs/last_page.png}
\frame[plain]{}

\end{document}